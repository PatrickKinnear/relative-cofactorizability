\documentclass{article}
\usepackage[utf8]{inputenc}

\usepackage{hyperref}
\usepackage{appendix}

\usepackage{quiver}
\usepackage{pkinne}
\usepackage{stmaryrd} % For mapsfrom
\usepackage{comment}
\usepackage{subcaption}

\usepackage{amsmath}
\usepackage{pict2e}

\makeatletter
\newcommand{\adjunction}{\@ifstar\named@adjunction\normal@adjunction}
\newcommand{\normal@adjunction}[4]{%
  % #1 : #2 <arrows> #3 : #4
  #1\colon #2%
  \mathrel{\vcenter{%
    \offinterlineskip\m@th
    \ialign{%
      \hfil$##$\hfil\cr
      \longrightharpoonup\cr
      \noalign{\kern-.3ex}
      \smallbot\cr
      \longleftharpoondown\cr
    }%
  }}%
  #3 \noloc #4%
}
\newcommand{\named@adjunction}[4]{%
  % #1 : #2 <arrows> #3 : #4
  #2%
  \mathrel{\vcenter{%
    \offinterlineskip\m@th
    \ialign{%
      \hfil$##$\hfil\cr
      \scriptstyle#1\cr
      \noalign{\kern.1ex}
      \longrightharpoonup\cr
      \noalign{\kern-.3ex}
      \smallbot\cr
      \longleftharpoondown\cr
      \scriptstyle#4\cr
    }%
  }}%
  #3%
}
\newcommand{\longrightharpoonup}{\relbar\joinrel\rightharpoonup}
\newcommand{\longleftharpoondown}{\leftharpoondown\joinrel\relbar}
\newcommand\noloc{%
  \nobreak
  \mspace{6mu plus 1mu}
  {:}
  \nonscript\mkern-\thinmuskip
  \mathpunct{}
  \mspace{2mu}
}
\newcommand{\smallbot}{%
  \begingroup\setlength\unitlength{.15em}%
  \begin{picture}(1,1)
  \roundcap
  \polyline(0,0)(1,0)
  \polyline(0.5,0)(0.5,1)
  \end{picture}%
  \endgroup
}
\makeatother

\usepackage{tikz}
\usepackage{ifthen}
\usepackage{intcalc}
\usetikzlibrary{positioning,calc}

\newcommand{\stackspace}{4}
\newcommand{\stackalternate}[2][1cm]{\;\tikz[baseline, yshift=.65ex]%
    {\foreach \k [evaluate=\k as \r using (.5*#2+.5-\k)*\stackspace] in {1,...,#2}{%
    \ifodd\k{\draw[->](0,\r pt)--(#1,\r pt);}%
    \else{\draw[<-](0,\r pt)--(#1,\r pt);}\fi    }}\;}
\newcommand{\stackrightarrow}[2][1cm]{\;\tikz[baseline, yshift=.65ex]%
    {\foreach \k [evaluate=\k as \r using (.5*#2+.5-\k)*\stackspace] in {1,...,#2}{%
    \draw[->](0,\r pt)--(#1,\r pt);%
    }}\;}
\newcommand{\stackleftarrow}[2][1cm]{\;\tikz[baseline, yshift=.65ex]%
    {\foreach \k [evaluate=\k as \r using (.5*#2+.5-\k)*\stackspace] in {1,...,#2}{%
    \draw[<-](0,\r pt)--(#1,\r pt);
    }}\;}

\title{Relative cofactorizability}
\author{Patrick Kinnear}
\date{}

\usepackage{biblatex}
\addbibresource{invertibility-conjecture.bib}

\newcommand\tens{\mathrm{tens}}
\def\RMod{\mathrm{RMod}}
\def\RCoMod{\mathrm{RCoMod}}
\def\LMod{\mathrm{LMod}}
\def\LCoMod{\mathrm{LCoMod}}
\def\Fun{\mathrm{Fun}}

\begin{document}

\maketitle

The invertibility conjecture is that $\Rep_q(G)$ is
invertible as a 1-morphism $\Rep(G) \to \Rep(G)$ in the Morita category $\Alg_3(\Pr)$.

The following theorem characterises invertible 1-morphisms in $\Alg_3(\Pr)$.

\begin{thm}
Let $\cA, \cB$ be $E_3$-algebras in $\cS$, and $\cC$ a morphism $\cA \to \cB$. That is, $\cC$ is an $E_2$-algebra in $(\cA, \cB)$-bimodules, or equivalently, an $E_2$-algebra in $\cS$ equipped with a braided tensor functor $\cA \bt \cB^{3 \op} \to Z_2(\cC)$. Then $\cC$ is invertible, considered as a morphism of $E_3$-algebras, if and only if it is dualizable as a module over $\cA, \cB, \cC^e_\cA := \cC^{\ot \op} \bt_\cA \cC, \cC^e_\cB := \cC \bt_\cB \cC^{\ot \op}, \cC^{\sigma \op} \bt_\cA \cC, \cC \bt_\cB \cC^{\sigma \op}, HC_\cA(\cC) := \cC \bt_{\cC^{\sigma \op} \bt_\cA \cC} \cC^{\ot \op}$, and $HC_\cB(\cC) := \cC \bt_{\cC \bt_\cB \cC^{\sigma \op}} \cC^{\ot \op}$ and moreover the following maps are isomorphisms:
\begin{enumerate}
        \item $HC_\cA(\cC) \to \Hom_\cB(\cC, \cC)$.
        \item $\cC^{\sigma \op} \bt_\cA \cC \to \Hom_{\cC^e_\cB}(\cC, \cC)$.
        \item $\cC \bt_\cB \cC^{\ot \op} \to \Hom_{\cC^{\sigma \op} \bt_\cA \cC}(\cC, \cC)$
        \item $\cB \to \Hom_{HC_\cA(\cC)}(\cC, \cC)$.
        \item $HC_\cB(\cC) \to \Hom_\cA(\cC, \cC)$.
        \item $\cA \to \Hom_{HC_\cB(\cC)}(\cC, \cC)$.
\end{enumerate}
\end{thm}

After \cite{brochierInvertibleBraidedTensor2020}, we refer to conditions 1 and 5 as relative cofactorizability, conditions 2 and 3 as relative factorizability, and conditions 4 and 6 as relative non-degeneracy. We are concerned here with relative factorizability.

We note that the functors in the conditions above are braided tensor functors \pk{!} so it suffices to show they are equivalences at the level of plain categories (we may ignore $\sigma \op, \ot \op$ etc).

\begin{notn}
There is a tensor product functor $T : \Rep_q(G) \bt \Rep_q(G) \to \Rep_q(G)$, and it has a right adjoint. We denote $\cO_q(G)^{FRT} = T^RT(\One) = \int^{X} X^{\vee} \bt X \in \Rep_q(G) \bt \Rep_q(G)$, and $\cO_q(G) = T(\cO_q^{FRT}(G)) = \int^{X} X^{\vee} \ot X \in \Rep_q(G)$. There are also corresponding objects $\cO^{FRT}(G), \cO(G)$ where the coend is now only over representations fo $G$/the image of Frobenius pullback/the M\"{u}ger center. Finally, there are objects $o^{FRT}_q(G) = \cO^{FRT}_q(G) \ot_{\cO^{FRT}(G)} \One, o_q(G) = \cO_q(G) \ot_{\cO(G)} \One$.
\end{notn}

\section{The proof}
We are interested in the functor
\[
HC_\cA(\cC) \to \Hom_\cA(\cC, \cC).
\]
The target consists of pairs $(F, \alpha)$, where $F : \cC \to \cC$ is a functor and $\alpha$ is a family of natural isomorphisms  $\alpha_a : F(a \ot - ) \to a \ot F(-)$ for $a \in \cA$. Morphisms $(F, \alpha) \to (G, \beta)$ in $\End_{\cA}(\cC)$ are given by natural transformations $\zeta : F \to G$ such that
% https://q.uiver.app/?q=WzAsNCxbMCwwLCJCIl0sWzIsMCwiQyJdLFswLDIsIkEiXSxbMiwyLCJEIl0sWzAsMV0sWzEsM10sWzIsM10sWzAsMl1d
\[\begin{tikzcd}
	F(a \ot - ) && G(a \ot - ) \\
	\\
	a \ot F(-) && a \ot G(-)
	\arrow["\zeta_{a \ot - }", from=1-1, to=1-3]
	\arrow["\beta_a", from=1-3, to=3-3]
	\arrow["a \ot \zeta", from=3-1, to=3-3]
	\arrow["\alpha_a", from=1-1, to=3-1]
\end{tikzcd}\]


The relative cofactorizability functor then takes
\[
c \bt d \mapsto (v \mapsto c \ot v \ot d, \sigma_{c, -}).
\]

Now, we have that there is an equivalence $HC_{\cA}(\cC) \simeq \LMod_{o_q(G)}(\cC)$. Moreover, there is an equivalence $\End_{\cA}(\cC) \simeq \LCoMod_{o_q(G)}(\cC)$. We would like to understand the functor below.

% https://q.uiver.app/?q=WzAsNCxbMCwwLCJCIl0sWzIsMCwiQyJdLFswLDIsIkEiXSxbMiwyLCJEIl0sWzAsMV0sWzIsMF0sWzEsM11d
\[\begin{tikzcd}
	HC_{\cA}(\cC) && \End_{\cA}(\cC) \\
	\\
	\LMod_{o_q(G)}(\cC) && \LCoMod_{o_q(G)}(\cC)
	\arrow[from=1-1, to=1-3]
	\arrow[from=3-1, to=1-1]
	\arrow[from=1-3, to=3-3]
\end{tikzcd}\]

We first try to understand the functor given by the first two legs, $\Phi : \LMod_{o_q(G)}(\cC) \to \End_{\cA}(\cC)$. We will make a claim for this functor and check that the composition $\Phi \circ \Psi$ is relative cofactorizability, where $\Psi : HC_{\cA}(\cC) \to \LMod_{o_q(G)}(\cC)$ implements the equivalence of the first leg.

\begin{claim}
The functor $\Psi$ is given on objects by
\[
c \bt d \mapsto o_q(G) \ot c \ot d.
\]
\end{claim}

\begin{proof}
We know that this functor breaks down as
\[
HC_{\cA}(\cC) = \cC \bt_{\cC^e_\cA} \cC \to \LMod_{\cO^{FRT}_q(G)}(\cC^e_\cA) \bt_{\cC^e_\cA} \cC \to \LMod^{\cC^e_{\cA}}_{\cO^{FRT}_q(G)}(\cC) \simeq \LMod^{\cC^e}_{o_q(G)}(\cC) \simeq \LMod_{o_q(G)}(\cC).
\]
The first arrow comes from a monadicity theorem, and sends
\[
c \bt d \mapsto (\int^x (x^{\vee} \ot c) \bt x) \bt d.
\]
The second arrow sends $m \boxtimes v \mapsto m \ot v$ (categorical action). and so the first two arrows send $c \ot d$ to $\int^x x^{\vee} \ot c \ot d \ot x$. The next equivalence simply lifts equivalence classes of the relative tensor product, but we have already abused notation here. The final equivalence exchanges the action via the enveloping algebra for the action internal to $\cC$, and so it maps $\int^x x^{\vee} \ot c \ot d \ot x = o_q^{FRT}(G) \ot (c \ot d)$ to $o_q(G) \ot c \ot d$.
\end{proof}

\begin{claim}
The functor $\Phi$ is given by
\[
M \mapsto (V \mapsto triv_l(V) \ot_{o_q(G)} M = V \ot M^{inv}, \alpha)
\]
where $\alpha$ is simply the associator.
\end{claim}

\begin{proof}
To check this, we would like to show that the composition of the equivalence $\Psi : HC_{\cA}(\cC) \to \LMod_{o_q(G)}(\cC)$ with $\Phi$ is relative cofactorizability.

We claimed above that $c \bt d$ maps to $o_q(G) \ot c \ot d$ under $\Psi$, that is, a free object. A morphism $f \bt g$ will become the morphism $1 \ot f \ot g$ in $\LMod_{o_q(G)}(\cC)$.

Under $\Phi$ the object $\o_q(G) \ot c \ot d$ becomes the map
\[
v \mapsto v \ot c \ot d.
\]

Note that morphisms $f: M \to N$ of $o_q(G)$-modules, will induce a morphism of diagrams
% https://q.uiver.app/?q=WzAsNixbMCwwLCJWIFxcb3Qgb19xKEcpIFxcb3QgTSJdLFsxLDAsIlYgXFxvdCBNIl0sWzIsMCwiViBcXG90X3tvX3EoRyl9IE0iXSxbMCwxLCJWIFxcb3Qgb19xKEcpIFxcb3QgTiJdLFsxLDEsIlYgXFxvdCBOIl0sWzIsMSwiViBcXG90X3tvX3EoRyl9IE4iXSxbMCwxLCJhY3RfTSIsMCx7Im9mZnNldCI6LTF9XSxbMSwyXSxbMCwzLCIxIFxcb3QgMSBcXG90IGYiLDJdLFszLDQsImFjdF9NIiwwLHsib2Zmc2V0IjotMX1dLFs0LDVdLFsyLDVdLFsxLDQsIjEgXFxvdCBmIiwyXSxbMCwxLCJhY3RfViIsMix7Im9mZnNldCI6MX1dLFszLDQsImFjdF9WIiwyLHsib2Zmc2V0IjoxfV1d
\[\begin{tikzcd}
	{V \ot o_q(G) \ot M} & {V \ot M} & {V \ot_{o_q(G)} M} \\
	{V \ot o_q(G) \ot N} & {V \ot N} & {V \ot_{o_q(G)} N}
	\arrow["{act_M}", shift left=1, from=1-1, to=1-2]
	\arrow[from=1-2, to=1-3]
	\arrow["{1 \ot 1 \ot f}"', from=1-1, to=2-1]
	\arrow["{act_M}", shift left=1, from=2-1, to=2-2]
	\arrow[from=2-2, to=2-3]
	\arrow[from=1-3, to=2-3]
	\arrow["{1 \ot f}"', from=1-2, to=2-2]
	\arrow["{act_V}"', shift right=1, from=1-1, to=1-2]
	\arrow["{act_V}"', shift right=1, from=2-1, to=2-2]
\end{tikzcd}\]
and hence a unique morphism of colimits. This defines the functor $\Phi$ on morphisms.

In particular, if $M = o_q(G) \ot M', N = o_q(G) \ot N'$ are free modules, then one can check that the top and bottom coequalizers are given by $act_V$. Then if $f = 1 \ot f'$ is in the image of the free $o_q(G)$-module functor, we have that the right vertical arrow is given by $1 \ot f'$.

Therefore, a map $f \bt g : c \bt d \to c' \bt d'$ in $HC_\cA(\cC)$ will become $1 \ot f \ot g$ in $\LMod_{o_q(G)}(\cC)$, and under $\Phi$ will become the natural transformation given by $1 \ot f \ot g$.

Then this composition of $\Phi \circ \Psi$ is identified with relative factorizability via the inverse braiding $\sigma^{-1}_{v, c} \ot 1 : v \ot c \ot d \to c \ot v \ot d$. That this is a natural transformation simply uses the fact that one can slide functions through braidings. It is easy to check that this is a morphism in $\End_{\cA}(\cC)$, hence up to natural isomorphism we have given the correct functor. \pk{Note that since $\cA$ is the M\"{u}ger centre, we could also have taken $\sigma_{v, c}$ here.}
\end{proof}

Now we would like to compose $\Phi$ with the equivalence $\End_{\cA}(\cC) \to \LCoMod_{o_q(G)}(\cC)$. This breaks down as a composition
\[
\End_{\cA}(\cC) \xrightarrow{A} \cC \bt_{\cA} \cC \xrightarrow{B} \LCoMod_{o_q(G)}(\cC).
\]

\begin{claim}
The functor $A$ is given by
\[
F \mapsto \int^x F(x^{\vee}) \bt x.
\]
\end{claim}

\begin{proof}
\pk{Complete this!}
\end{proof}

\begin{claim}
The functor $B$ is given on underlying objects by $T : \cC \bt_{\cA} \cC \to \cC$, the relative tensor product.
\end{claim}

\begin{proof}
The functor $B$ comes from the comonadicity theorem, applied to the comonad induced by $T \dashv T^R$ where $T : \cC \bt_\cA \cC \to \cC$ is te ``relative tensor product". Then the comparison functor is, on underlying objects, simply $T$. The comodule structure will be the free comodule structure coming from the unit for $o_q(G)$ as a bialgebra, i.e. $M \xrightarrow{\eta \ot 1} o_q(G) \ot M$.
\end{proof}

We note that the above comodule strucutre factors through the free $o_q^{FRT}(G)$-comodule strucutre, in the usual way.

Putting together everything so far, we have a functor $B \circ A \circ \Phi : \LMod_{o_q(G)}(\cC) \to \LCoMod_{o_q(G)}(\cC)$ which takes $M$ to the object $o_q(G) \ot M^{inv}$ with the free comodule structure. In particular, we send free modules to free comodules, i.e. the diagram of Fig. \ref{(co)free-preserving-functor} commutes.

Now, such a functor $F$ is specified by a certain pairing on $A$. As detailed in Appendix \ref{app-comod-mod-pairings}, the pairing is given by $\epsilon \circ F(m)$, and $F$ is an equivalence if and only if this pairing is nondegenerate.

Therefore, we must ask where the map of $A$-modules $A \ot A \to A$ that is multiplication is sent, and it will then give us a pairing on postcomposing with the counit $\epsilon$ which we will show is nondegenerate.

\begin{claim}
The functor $\Phi$ sends the multiplication map $A \ot A \to A$ to the natural transformation given by the right-action $V \ot A \to V$, mapping $V \ot - \to \Id$.
\end{claim}

\begin{proof}
We know that the natural transofrmation will be given by the unique vertical map making the diagram
\[\begin{tikzcd}
	{V \ot o_q(G) \ot o_q(G) \ot o_q(G)} & {V \ot o_q(G) \ot o_q(G)} & {V \ot o_q(G)} \\
	{V \ot o_q(G) \ot o_q(G)} & {V \ot o_q(G)} & {V}
	\arrow["{1 \ot m \ot 1}", shift left=1, from=1-1, to=1-2]
	\arrow[from=1-2, to=1-3]
	\arrow["{1 \ot 1 \ot m}"', from=1-1, to=2-1]
	\arrow["{1 \ot m}", shift left=1, from=2-1, to=2-2]
	\arrow[from=2-2, to=2-3]
	\arrow[from=1-3, to=2-3]
	\arrow["{1 \ot m}"', from=1-2, to=2-2]
	\arrow["{act_V \ot 1 \ot 1}"', shift right=1, from=1-1, to=1-2]
	\arrow["{act_V}"', shift right=1, from=2-1, to=2-2]
\end{tikzcd}\]
commute.
Consider the bottom row of this diagram. We claim that the map in this coequalizer is $act_V$. It is clear that this is a cofork. Moreover, given any cofork map $\phi : V \ot o_q(G) \to V$, this factors uniquely as $\tilde{\phi} \circ act_V$, where $\tilde{\phi} : V \to V : v \mapsto \phi(v \ot 1)$. So $act_V$ has the universal property.
Similarly, the coequalizer for the top row of the diagram is $act_V \ot 1$.
Now, by the definition of what it means to have a module structure, putting the map $act_V$ at the right vertical arrow makes the diagram commute. By uniqueness of the maps induced under colimits, this must be the image of the multiplication map under $\Phi$.
\end{proof}

Now, under the functors $B \circ A$, this involves putting maps on the left leg of a copy of $o_q(G)$. \pk{!?} Then recall that the right action on a trivial module is given diagramatically by \pk{DIAGRAM}, and so the pairing on $o_q(G)$ will be given by \pk{DIAGRAM} on correcting for FRT vs left/right versions of things \pk{!}. But we have already shown this pairing to be nondegenerate!

This concludes the proof of relative cofactorizability.

TODO:
\begin{enumerate}
\item Check the pairing-functor equivalence only depends on free objects; $F(m \ot 1) = F(m) \ot 1$.
\item Complete claim 1.3, and check what happens for morphisms. See \cite[Thm. 1.11]{kalmykovCategoricalApproachDynamical2020} and \cite{hoyoisCategorifiedGrothendieckRiemannRochTheorem2020}. Should manage to go directly by establishing self-duality $\cA$-relatively, if you do it right, no need to factor through Gaitsgory's appendix.
\item Go over the diagrams carefully and write it up.
\end{enumerate}

\begin{appendices}

\section{Comodules and modules}
\label{app-comod-mod-pairings}

Let $\cC$ be a monoidal category, $A$ an algebra objectand $C$ a coalgebra object. Suppose there is a functor $F : \LMod_A(\cC) \to \LCoMod_c(\cC)$ making the diagram in Fig. \ref{(co)free-preserving-functor} commute.
\begin{figure}
\centering
% https://q.uiver.app/?q=WzAsMyxbMCwwLCJcXExNb2RfQShcXGNDKSJdLFsyLDAsIlxcTENvTW9kX0EoXFxjQykiXSxbMSwxLCJcXGNDIl0sWzAsMV0sWzIsMCwiZnJlZSJdLFsyLDEsImZyZWUiLDJdXQ==
\begin{tikzcd}
	{\LMod_A(\cC)} && {\LCoMod_A(\cC)} \\
	& \cC
	\arrow[from=1-1, to=1-3]
	\arrow["free", from=2-2, to=1-1]
	\arrow["cofree"', from=2-2, to=1-3]
\end{tikzcd}
\caption{Functors of interest.}
\label{(co)free-preserving-functor}
\end{figure}

We will show here that such functors are in correspondence with pairings $\Omega : C \ot A \to \One$.

First, recall that a free module has the property that
\begin{align*}
\Hom_A(A \ot M, X) &\cong \Hom(M, X)\\
f &\mapsto f \circ (\eta \ot 1)\\
act_X \circ (1 \ot g) &\mapsfrom g
\end{align*}
and a cofree comodule has the property that
\begin{align*}
\Hom_C(Y, C \ot N) &\cong \Hom(Y, N)\\
f &\mapsto (\epsilon \ot 1) \circ f\\
(1 \ot g) \circ coact_Y &\mapsfrom g.
\end{align*}

Now, to obtain a pairing from a functor $F$, we use the image of the identity map $A \to A \cong A \ot \One$ under the composite
\[
\Hom(A, A) \cong \Hom_{A}(A \ot A, A \ot \One) \xrightarrow{F} \Hom_C(C \ot A, C \ot \One) \cong \Hom(C \ot A, \One).
\]
So $F$ produces the pairing $\epsilon \circ F(m)$.

Conversely, given a pairing $\Omega : C \ot A \to \One$, this induces a functor on free modules given by
\begin{align*}
\Hom_A(A \ot M, A \ot N) \to \Hom(M, A \ot N) \to \Hom(C \ot M, N) \to \Hom_C(C \ot M, C \ot N)\\
f \mapsto f \circ (\eta \ot 1) \mapsto \Omega \circ (1 \ot f \circ (\eta \ot 1)) \mapsto (1 \ot \Omega \circ (1 \ot f \circ (\eta \ot 1))) \circ coact_{C \ot M}
\end{align*}

Now, let us begin with $\Omega$, induce a functor, and then induce a pairing. The resulting pairing is
\[
(\epsilon \ot 1) \circ (1 \ot \Omega \circ (1 \ot m \circ (\eta \ot 1))) \circ coact_A = (\epsilon \ot 1) \circ (1 \ot \Omega) \circ coact_{C \ot A}.
\]
Notice that if the pairing $\Omega$ has the property that
\[
(\epsilon \ot 1) \circ (1 \ot \Omega) \circ coact_{C \ot A} = \Omega
\]
then we are done.

It remains to check that, given a functor $F$, then when we produce its associated copairing has the desired commutation property, and that the associated functor of this is $F$.

The commutation property follows since $F$ takes free modules to free comodules, so sends $m$, a map of free modules, to a map $F(m)$ of free comodules. Then
\[
\epsilon \circ \epsilon \circ F(m) \circ coact_{C \ot A} = \epsilon \circ \epsilon \circ coact_C \circ F(m) = \epsilon \circ F(m)
\]
as desired.

We \pk{claim} that a functor is specified by what it does on free modules, indeed on $A$, so it suffices to understand this.

So what is the functor induced by $\epsilon \circ F(m)$? On free objects it takes $f : A \ot M \to A \ot N$ to the map
\[
C \ot M \xrightarrow{coact_{C \ot M}} C \ot C \ot M \xrightarrow{1 \ot 1 \ot \eta \ot 1} C \ot C \ot A \ot M \xrightarrow{1 \ot 1 \ot f} C \ot C \ot A \ot N \xrightarrow{1 \ot F(m) \ot 1} C \ot C \ot N \xrightarrow{1 \ot \epsilon} C \ot N.
\]
Notice that for $M = A, N = \One, f = m$ then the chain of inner arrows is just $F(m)$, and moreover this is a comodule map so that
\[
\epsilon \circ F(m) \circ coact_{C \ot A} = \epsilon \circ coact_C \circ F(m) = F(m).
\]
In other words, the functor induced by the pairing induced by $F$ sends $m$ to $F(m)$.

More generally, let $f : A \ot M \to A \ot N$ be a map of free $A$-modules. Notice that the composition
% https://q.uiver.app/?q=WzAsNSxbMCwwLCJBIFxcb3QgTSJdLFsxLDAsIkEgXFxvdCBBIFxcb3QgTSJdLFsyLDAsIkEgXFxvdCBNIl0sWzIsMSwiQSBcXG90IE4iXSxbMSwxLCJBIFxcb3QgQSBcXG90IE4iXSxbMCwxLCIxIFxcb3QgXFxldGEgXFxvdCAxIl0sWzEsMiwibSBcXG90IDEiXSxbMiwzLCJmIl0sWzEsNCwiMSBcXG90IGYiLDJdLFs0LDMsIm0gXFxvdCAxIiwyXV0=
\[\begin{tikzcd}
	{A \ot M} & {A \ot A \ot M} & {A \ot M} \\
	& {A \ot A \ot N} & {A \ot N}
	\arrow["{1 \ot \eta \ot 1}", from=1-1, to=1-2]
	\arrow["{m \ot 1}", from=1-2, to=1-3]
	\arrow["f", from=1-3, to=2-3]
	\arrow["{1 \ot f}"', from=1-2, to=2-2]
	\arrow["{m \ot 1}"', from=2-2, to=2-3]
\end{tikzcd}\]
is simply $f$. Also note that, since $F$ makes the diagram in Fig. \ref{(co)free-preserving-functor} commute, and since the (co)free (co)module functor sends $M \xrightarrow{g} N$ to $1 \ot g$, we have that $F(1 \ot g) = 1 \ot g$, so that applying $F$ to the above diagram gives that $F(f)$ factors as follows.
\[\begin{tikzcd}
	{C \ot M} & {C \ot A \ot M} & {C \ot M} \\
	& {C \ot A \ot N} & {C \ot N}
	\arrow["{1 \ot \eta \ot 1}", from=1-1, to=1-2]
	\arrow["{F(m \ot 1)}", from=1-2, to=1-3]
	\arrow["F(f)", from=1-3, to=2-3]
	\arrow["{1 \ot f}"', from=1-2, to=2-2]
	\arrow["{F(m \ot 1)}"', from=2-2, to=2-3]
\end{tikzcd}\]
Then, if $F(m \ot 1) = F(m) \ot 1$, this establishes that the functor is recovered on free modules, \pk{hence on all modules}.

We have therefore shown that there is a 1:1 correspondence
\begin{equation*}
        \left\{\begin{array}{c}
        \text{Pairings}\\
        \text{$\Omega : C \ot A \to \One$ s.t.}\\
        (\epsilon \ot 1) \circ (1 \ot \Omega) \circ coact_{C \ot A} = \Omega
        \end{array}\right\}
        \xleftrightarrow{1:1}
        \left\{\begin{array}{c}
        \text{Functors $\LMod_A(\cC) \to \LCoMod_C(\cC)$}\\
        \text{such that the diagram in}\\
        \text{Fig. \ref{(co)free-preserving-functor} commutes.}
        \end{array}\right\}.
\end{equation*}

Now consider the situation where $A$ and $C$ are finite-dimenstional. Then we claim that the functor associated to a pairing $\Omega$ is an equivalence if and only if $\Omega$ is nondegenerate.

Since the functor makes the diagram in Figure \ref{(co)free-preserving-functor} commute, it \pk{must be} essentially surjective. To check that it is fully faithful it suffices to check that the function
\begin{align*}
\Hom(M, A \ot N) &\to \Hom(C \ot M, N)\\
f &\mapsto (\Omega \ot 1) \circ (1 \ot f)
\end{align*}
is a bijection. We notice that this factors as
\begin{align*}
\Hom(M, A \ot N) &\to \Hom(M, C^* \ot N) &\xrightarrow{\cong} \Hom(C \ot M, N)\\
f &\mapsto (\omega \ot 1) \circ f &\mapsto (ev \ot 1) \circ (1 \ot ((\omega \ot 1) \circ f))
\end{align*}
where $\omega : A \to C^*$ is the homomorphism induced by $\Omega$, with the property that $(ev \ot 1) \circ (1 \ot \omega) = \Omega$. Then clearly the above function is a bijection if and only if $\omega$ is an isomorphism, which is true if and only if $\Omega$ is non-degenerate.

\end{appendices}

\printbibliography

\end{document}
