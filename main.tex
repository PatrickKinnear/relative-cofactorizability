\documentclass{article}
\usepackage[utf8]{inputenc}

\usepackage{hyperref}
\usepackage{appendix}

\usepackage{quiver}
\usepackage{pkinne}
\usepackage{stmaryrd} % For mapsfrom
\usepackage{comment}

\usepackage{amsmath}
\usepackage{pict2e}

\makeatletter
\newcommand{\adjunction}{\@ifstar\named@adjunction\normal@adjunction}
\newcommand{\normal@adjunction}[4]{%
  % #1 : #2 <arrows> #3 : #4
  #1\colon #2%
  \mathrel{\vcenter{%
    \offinterlineskip\m@th
    \ialign{%
      \hfil$##$\hfil\cr
      \longrightharpoonup\cr
      \noalign{\kern-.3ex}
      \smallbot\cr
      \longleftharpoondown\cr
    }%
  }}%
  #3 \noloc #4%
}
\newcommand{\named@adjunction}[4]{%
  % #1 : #2 <arrows> #3 : #4
  #2%
  \mathrel{\vcenter{%
    \offinterlineskip\m@th
    \ialign{%
      \hfil$##$\hfil\cr
      \scriptstyle#1\cr
      \noalign{\kern.1ex}
      \longrightharpoonup\cr
      \noalign{\kern-.3ex}
      \smallbot\cr
      \longleftharpoondown\cr
      \scriptstyle#4\cr
    }%
  }}%
  #3%
}
\newcommand{\longrightharpoonup}{\relbar\joinrel\rightharpoonup}
\newcommand{\longleftharpoondown}{\leftharpoondown\joinrel\relbar}
\newcommand\noloc{%
  \nobreak
  \mspace{6mu plus 1mu}
  {:}
  \nonscript\mkern-\thinmuskip
  \mathpunct{}
  \mspace{2mu}
}
\newcommand{\smallbot}{%
  \begingroup\setlength\unitlength{.15em}%
  \begin{picture}(1,1)
  \roundcap
  \polyline(0,0)(1,0)
  \polyline(0.5,0)(0.5,1)
  \end{picture}%
  \endgroup
}
\makeatother

\usepackage{tikz}
\usepackage{ifthen}
\usepackage{intcalc}
\usetikzlibrary{positioning,calc}

\newcommand{\stackspace}{4}
\newcommand{\stackalternate}[2][1cm]{\;\tikz[baseline, yshift=.65ex]%
    {\foreach \k [evaluate=\k as \r using (.5*#2+.5-\k)*\stackspace] in {1,...,#2}{%
    \ifodd\k{\draw[->](0,\r pt)--(#1,\r pt);}%
    \else{\draw[<-](0,\r pt)--(#1,\r pt);}\fi    }}\;}
\newcommand{\stackrightarrow}[2][1cm]{\;\tikz[baseline, yshift=.65ex]%
    {\foreach \k [evaluate=\k as \r using (.5*#2+.5-\k)*\stackspace] in {1,...,#2}{%
    \draw[->](0,\r pt)--(#1,\r pt);%
    }}\;}
\newcommand{\stackleftarrow}[2][1cm]{\;\tikz[baseline, yshift=.65ex]%
    {\foreach \k [evaluate=\k as \r using (.5*#2+.5-\k)*\stackspace] in {1,...,#2}{%
    \draw[<-](0,\r pt)--(#1,\r pt);
    }}\;}

\title{Relative factorizability}
\author{Patrick Kinnear}
\date{}

\usepackage{biblatex}
\addbibresource{invertibility-conjecture.bib}

\newcommand\tens{\mathrm{tens}}
\def\RMod{\mathrm{RMod}}
\def\RCoMod{\mathrm{RCoMod}}
\def\LMod{\mathrm{LMod}}
\def\LCoMod{\mathrm{LCoMod}}
\def\Fun{\mathrm{Fun}}

\begin{document}

\maketitle

The invertibility conjecture is that $\Rep_q(G)$ is
invertible as a 1-morphism $\Rep(G) \to \Rep(G)$ in the Morita category $\Alg_3(\Pr)$.

The following theorem characterises invertible 1-morphisms in $\Alg_3(\Pr)$.

\begin{thm}
Let $\cA, \cB$ be $E_3$-algebras in $\cS$, and $\cC$ a morphism $\cA \to \cB$. That is, $\cC$ is an $E_2$-algebra in $(\cA, \cB)$-bimodules, or equivalently, an $E_2$-algebra in $\cS$ equipped with a braided tensor functor $\cA \bt \cB^{3 \op} \to Z_2(\cC)$. Then $\cC$ is invertible, considered as a morphism of $E_3$-algebras, if and only if it is dualizable as a module over $\cA, \cB, \cC^e_\cA := \cC^{\ot \op} \bt_\cA \cC, \cC^e_\cB := \cC \bt_\cB \cC^{\ot \op}, \cC^{\sigma \op} \bt_\cA \cC, \cC \bt_\cB \cC^{\sigma \op}, HC_\cA(\cC) := \cC \bt_{\cC^{\sigma \op} \bt_\cA \cC} \cC^{\ot \op}$, and $HC_\cB(\cC) := \cC \bt_{\cC \bt_\cB \cC^{\sigma \op}} \cC^{\ot \op}$ and moreover the following maps are isomorphisms:
\begin{enumerate}
        \item $HC_\cA(\cC) \to \Hom_\cB(\cC, \cC)$.
        \item $\cC^{\sigma \op} \bt_\cA \cC \to \Hom_{\cC^e_\cB}(\cC, \cC)$.
        \item $\cC \bt_\cB \cC^{\ot \op} \to \Hom_{\cC^{\sigma \op} \bt_\cA \cC}(\cC, \cC)$
        \item $\cB \to \Hom_{HC_\cA(\cC)}(\cC, \cC)$.
        \item $HC_\cB(\cC) \to \Hom_\cA(\cC, \cC)$.
        \item $\cA \to \Hom_{HC_\cB(\cC)}(\cC, \cC)$.
\end{enumerate}
\end{thm}

After \cite{brochierInvertibleBraidedTensor2020}, we refer to conditions 1 and 5 as relative cofactorizability, conditions 2 and 3 as relative factorizability, and conditions 4 and 6 as relative non-degeneracy. We are concerned here with relative factorizability.

We note that the functors in the conditions above are braided tensor functors \pk{!} so it suffices to show they are equivalences at the level of plain categories (we may ignore $\sigma \op, \ot \op$ etc).

\begin{notn}
There is a tensor product functor $T : \Rep_q(G) \bt \Rep_q(G) \to \Rep_q(G)$, and it has a right adjoint. We denote $\cO_q(G)^{FRT} = T^RT(\One) = \int^{X} X^{\vee} \bt X \in \Rep_q(G) \bt \Rep_q(G)$, and $\cO_q(G) = T(\cO_q^{FRT}(G)) = \int^{X} X^{\vee} \ot X \in \Rep_q(G)$. There are also corresponding objects $\cO^{FRT}(G), \cO(G)$ where the coend is now only over representations fo $G$/the image of Frobenius pullback/the M\"{u}ger center. Finally, there are objects $o^{FRT}_q(G) = \cO^{FRT}_q(G) \ot_{\cO^{FRT}(G)} \One, o_q(G) = \cO_q(G) \ot_{\cO(G)} \One$.
\end{notn}

\section{Sketch of the strategy}
We are interested in showing the functor
\[
\Rep_q(G) \bt_{\Rep(G)} \Rep_q(G) \to \End_{\Rep_q(G) \bt_{\Rep(G)} \Rep_q(G)}(\Rep_q(G))
\]
given on objects by acting on the left and right, is an equivalence.

To show this, we will proceed as follows.

\begin{enumerate}
\item We will show that $\Rep_q(G) \bt_{\Rep(G)} \Rep_q(G) \simeq \RCoMod_{o_q(G)}(\Rep_q(G))$, and that $\End_{\Rep_q(G) \bt_{\Rep(G)} \Rep_q(G)}(\Rep_q(G)) \simeq \RMod_{o_q(G)}(\Rep_q(G))$. Then it suffices to consider the induced functor $\RCoMod_{o_q(G)}(\Rep_q(G))\to \RMod_{o_q(G)}(\Rep_q(G))$ and show this is an equivalence.
\item We recall that a functor $\RCoMod_{A}(\cC)\to \RMod_{A}(\cC))$ is equivalent to a Hopf pairing $\Omega : A \ot A \to \One$, and that the functor is an equivalence if and only if the pairing is non-degenerate. The functor $F : \RCoMod_{A}(\cC)\to \RMod_{A}(\cC)$ corresponds to the pairing
\[
A \ot A \xrightarrow{\nabla_F} A \xrightarrow{\epsilon} \One
\]
where $\nabla_F$ is the module structure obtained under $F$ from the standard comodule structure on $A$. See Appendix \ref{app-comod-mod-pairings} for details.
\item We will compute the pairing on $o_q(G)$ coming from our functor $\RCoMod_{o_q(G)}(\Rep_q(G))\to \RMod_{o_q(G)}(\Rep_q(G))$. We will show that, under a forgetful functor $\Rep_q(G) \to \Rep u_q(G)$, this pairing is one which is known to be non-degenerate.
\item To compute the pairing, we will first compute a pairing on $\cO_q(G)$ coming from the functor $\RCoMod_{\cO_q(G)}(\Rep_q(G))\to \RMod_{\cO_q(G)}(\Rep_q(G))$ induced by the non-relative version of the factorizability functor.
\item Then, we will show that this pairing on $\cO_q(G)$ descends to the pairing on $o_q(G)$, and it will be clear from the $\cO_q(G)$ computations that the pairing is the one we want under the forgetful functor.
\end{enumerate}

Therefore, we must first treat the non-relative case of the factorizability functor and obtain the corresponding pairing. The multiplication for $\cO_q(G)$ and the desured pairing are shown in Figure \ref{f-mult-pairing}, where diagrams are read bottom-top.

\begin{figure}
\centering
\includegraphics[width=\textwidth]{img/mult-pairing.png}
\caption{The multiplication and Hopf pairing.}
\label{f-mult-pairing}
\end{figure}

\section{The non-relative case}

In this case, we have the following.
\begin{enumerate}
\item Using cp-rigidity of $\Rep_q(G)$, we have that the tensor product functor $\Rep_q(G) \bt \Rep_q(G) \to \Rep_q(G)$ has a right adjoint, $X \mapsto (X \bt \One) \ot \cO^{FRT}_q(G)$, and this adjunction is monadic. Hence the comparison functor
\[
X \mapsto (X \bt \One) \ot \cO^{FRT}_q(G)
\]
is an equivalence of categories $\Rep_q(G) \simeq \RMod_{\cO_q^{FRT}(G)}(\Rep_q(G) \bt \Rep_q(G))$. We have given a functor going one way: the functor going the other will be taking $\cO_q(G)$-coinvariants, and this can be seen by looking in detail at the proof of the monadicity theorem.

In other words, the inverse of the comparison functor will take $\cO^{FRT}_q(G)$ to $\One$. The reason is that, for $T, T^R$ the adjunction, the functor back from $T^RT$-algebras will take a $T^RT$-algebra to the coequalizer of the diagram
% https://q.uiver.app/?q=WzAsMixbMCwwLCJUVF5SVEEiXSxbMiwwLCJUQSJdLFswLDEsIlxcZXBzaWxvbl97VEF9IiwyLHsib2Zmc2V0IjoyfV0sWzAsMSwiVFxcYWxwaGEiLDAseyJvZmZzZXQiOi0yfV1d
\[\begin{tikzcd}
	{TT^RTA} && TA
	\arrow["{\epsilon_{TA}}"', shift right=2, from=1-1, to=1-3]
	\arrow["T\alpha", shift left=2, from=1-1, to=1-3]
\end{tikzcd}\]
but when $A$ is itself of the form $T^RT(\One)$ then it is easy to see that $TT^RT(\One) \xrightarrow{\epsilon_{T\One}} T(\One)$ equalizes, since in this case $\alpha = T^R\epsilon_{T(\One)}$. See \cite[Lemma 5.10]{johnstoneCategoryTheoryLecture2019} for more details.

\item We then have a functor $\Rep_q(G) \bt \Rep_q(G) \to \RMod_{\cO_q^{FRT}(G)}(\Rep_q(G) \bt \Rep_q(G))$. It is composed of $\ot$ and the above monadic equivalence. So in particular it has a right adjoint. The functor, up to a natural transformation, is given by
\[
X \bt Y \mapsto (X \bt Y) \ot \cO^{FRT}_q(G)
\]
i.e. it is the free right module functor. Now, we regard any object in $\Rep_q(G) \bt \Rep_q(G)$ as a right module over its unit $\One$, i.e. $\Rep_q(G) \bt \Rep_q(G) \simeq \RMod_{\One}(Rep_q(G) \bt \Rep_q(G))$. Then the functor in question is $- \ot_{\One} \cO_q(G)$, tensoring along the inclusion $\One \to \cO_q(G)$. We regard $\One$ as a trivial, and hence trivially commutative, Hopf algebra. Then by \cite[Prop. 3.12]{arkhipovAnotherRealizationCategory2002} $\cO_q(G)$ is a faithfully flat module, which is to say that the above functor is comonadic. So we have
\[
\Rep_q(G) \bt \Rep_q(G) \simeq \RCoMod_{\cO_q^{FRT}(G)}(\RMod_{\cO_q^{FRT}(G)}(\Rep_q(G) \bt \Rep_q(G))).
\]
Again, the functor right-left is given by $-\ot_{\cO_q^{FRT}(G)} \One$.


There is also an equivalence $\RCoMod_{\cO_q(G)}(\Rep_q(G)) \to \RCoMod_{\cO^{FRT}_q(G)}(\Rep_q(G))$. The source category is really the category of modules for an algebra $\cO_q(G) = T(\cO_q^{FRT}(G))$ internal to $\Rep_q(G)$, while the target category is a category of module objects defined via the action of $\Rep_q(G) \bt \Rep_q(G)$, and they are related by braiding (see a later point).

So by now we have a composition
\begin{align*}
\RCoMod_{\cO_q(G)}(\Rep_q(G)) &\xrightarrow{\sim} \RCoMod_{\cO^{FRT}_q(G)}(\RMod_{\cO^{FRT}_q(G)}(\Rep_q(G) \bt \Rep_q(G)))\\ &\xrightarrow{\sim} \Rep_q(G) \bt \Rep_q(G)\\
\end{align*}
sending $\cO_q(G) \mapsto \cO^{FRT}_q(G) \bt \cO^{FRT}_q(G) \mapsto \cO^{FRT}_q(G)$.

\item We also have that the target of factorizability can be written as
\begin{align*}
&\Fun_{\Rep_q(G) \bt \Rep_q(G)}(\Rep_q(G), \Rep_q(G))\\
&\xrightarrow{\sim} \Fun_{\Rep_q(G) \bt \Rep_q(G)}(\LMod_{\cO^{FRT}_q(G)}(\Rep_q(G) \bt \Rep_q(G)), \Rep_q(G))\\
&\xrightarrow{\sim} \RMod_{\cO^{FRT}_q(G)}(\Rep_q(G)).
\end{align*}
The first map is induced by the free $\cO_q^{FRT}(G)$-module functor $\Rep_q(G) \to \LMod_{\cO^{FRT}_q(G)}(\Rep_q(G)$, so it sends $F \mapsto F \circ \One \ot_{\cO_q^{FRT}(G)} -$. The second sends $F \mapsto F(A)$, so that in total we have a functor $F \mapsto F(\One)$.

We need to understand the module structure here.

We then have, similar to before, that
\begin{equation}
\label{eq-FRT-ordinary-modules}
\RMod_{\cO^{FRT}_q(G)}(\Rep_q(G)) \simeq \RMod_{\cO_q(G)}(\Rep_q(G)).
\end{equation}

Putting this together, we get a functor
\begin{align*}
\End_{\Rep_q(G) \bt \Rep_q(G)}(\Rep_q(G)) \to \RMod_{\cO_q(G)}(\Rep_q(G)).
\end{align*}

\item Now, we've seen that $\cO_q(G) \mapsto \cO_q^{FRT}(G) \in \Rep_q(G) \bt \Rep_q(G)$. Then under the factorizability map, this becomes the usual action of $\cO^{FRT}_q(G)$ by left/right multiplication. Then, under the functors above, we should obtain a module structure on $\cO_q(G)$, and it had better be the pairing $\Omega$ once we compose with $\epsilon$. To see this, it effectively suffices to understand how the map equiavelnce (\ref{eq-FRT-ordinary-modules}) works.

\item In general, let $M$ be a module in the sense of the action of $\Rep_q(G) \bt \Rep_q(G)$ on $\Rep_q(G)$, for $\cO_q^{FRT}(G)$. So there will be maps $X^{\vee} \ot M \ot X \xrightarrow{\nabla} M$. Then one can check that $M$ has the structure of a right module for $\cO_q^{FRT}(G)$ under the action
\begin{align*}
M \ot X^{\vee} \ot X \xrightarrow{T(\nabla) \circ \sigma_{M, X^{\vee}}} M.
\end{align*}
And conversely, precompsing the maps for an action internal to $\Rep_q(G)$ with $\sigma_{M, X^{\vee}}^{-1}$ gives a functor the other way.

\begin{figure}
\centering
\includegraphics[width=\textwidth]{img/module-equiv}
\caption{The equivalence of modules for $\cO^{FRT}_q(G)$ and $\cO_q(G)$.}
\label{f-module-equiv}
\end{figure}

Then, under this identification, we are almost there: just need to figure out how to write the self-action for $\cO^{FRT}_q(G)$ appropriately.
\item There are two ways to interpret this. One is that under the above equivalence, the multiplication on $\cO_q^{FRT(G)}$ should be obtained from that on $\cO_q(G)$. Working this out, we'd then have the pairing of Figure \\ref{f-degen-pairing}.

\begin{figure}
\centering
\includegraphics[width=0.5\textwidth]{img/degen-pairing}
\caption{The pairing suggested by the equivalence of internal and $FRT$ structures.}
\label{f-degen-pairing}
\end{figure}

On the other hand, we recall that we're regarding $\cO_q(G)$ as $F(\One)$, where $F$ is the endofunctor for $\Rep_q(G)$ given by acting with $\cO_q(G)$. There is a clear way of giving this a module structure:
\[
F(\One) \ot \cO_q(G) \to F(\cO_q(G)) \xrightarrow{F(\epsilon)} F(\One)
\]
which again gives the same pairing, see Fig. \ref{f-degen-pairing-2}.

\begin{figure}
\centering
\includegraphics[width=0.5\textwidth]{img/degen-pairing-2}
\caption{The pairing suggested by using the counit.}
\label{f-degen-pairing-2}
\end{figure}

\item However, this is not the pairing we want. Instead, we'd like the pairing shown in Fig. \ref{f-desired-pairing}.

\begin{figure}
\centering
\includegraphics[width=\textwidth]{img/desired-pairing}
\caption{The pairing we would like suggests a particular form of multiplication.}
\label{f-desired-pairing}
\end{figure}

And so, we'd like to say something like, multiplication should come from a map shown in the figure.

However both source and target in this diagram are simply parts of the coend $\cO_q(G)$ (and not its product...).
\end{enumerate}

\section{The relative case}

Need to fill this in: check that the pairing from the last section descends to $o_q(G)$ in the relative case.

\begin{appendices}

\section{Comodules and modules}
\label{app-comod-mod-pairings}

Let $\cC$ be a monoidal category, and $A$ a bialgebra object. Let $\Omega : A \ot A \to \One$ be a Hopf pairing, that is, a pairing such that the following diagrams commute.
% https://q.uiver.app/?q=WzAsNSxbMCwwLCJBIFxcb3QgQSBcXG90IEEiXSxbNCwwLCJBIFxcb3QgQSJdLFswLDIsIkEgXFxvdCBBIFxcb3QgQSBcXG90IEEiXSxbMiwyLCJBIFxcb3QgQSJdLFs0LDIsIkEiXSxbMCwxLCJcXG5hYmxhIFxcb3QgMSJdLFswLDIsIjEgXFxvdCAxIFxcb3QgXFxEZWx0YSIsMl0sWzIsMywiMSBcXG90IFxcT21lZ2EgXFxvdCAxIiwyXSxbMyw0LCJcXE9tZWdhIiwyXSxbMSw0LCJcXE9tZWdhIl1d
\[\begin{tikzcd}
	{A \ot A \ot A} &&&& {A \ot A} \\
	\\
	{A \ot A \ot A \ot A} && {A \ot A} && \One
	\arrow["{\nabla \ot 1}", from=1-1, to=1-5]
	\arrow["{1 \ot 1 \ot \Delta}"', from=1-1, to=3-1]
	\arrow["{1 \ot \Omega \ot 1}"', from=3-1, to=3-3]
	\arrow["\Omega"', from=3-3, to=3-5]
	\arrow["\Omega", from=1-5, to=3-5]
\end{tikzcd}\]

% https://q.uiver.app/?q=WzAsNSxbMCwwLCJBIFxcb3QgQSBcXG90IEEiXSxbNCwwLCJBIFxcb3QgQSJdLFswLDIsIkEgXFxvdCBBIFxcb3QgQSBcXG90IEEiXSxbMiwyLCJBIFxcb3QgQSJdLFs0LDIsIkEiXSxbMCwxLCJcXG5hYmxhIFxcb3QgMSJdLFswLDIsIjEgXFxvdCAxIFxcb3QgXFxEZWx0YSIsMl0sWzIsMywiMSBcXG90IFxcT21lZ2EgXFxvdCAxIiwyXSxbMyw0LCJcXE9tZWdhIiwyXSxbMSw0LCJcXE9tZWdhIl1d
\[\begin{tikzcd}
	{A \ot A \ot A} &&&& {A \ot A} \\
	\\
	{A \ot A \ot A \ot A} && {A \ot A} && \One
	\arrow["{1 \ot \nabla}", from=1-1, to=1-5]
	\arrow["{\Delta \ot 1 \ot 1}"', from=1-1, to=3-1]
	\arrow["{1 \ot \Omega \ot 1}"', from=3-1, to=3-3]
	\arrow["\Omega"', from=3-3, to=3-5]
	\arrow["\Omega", from=1-5, to=3-5]
\end{tikzcd}\]

% https://q.uiver.app/?q=WzAsNCxbMCwwLCJBIl0sWzIsMCwiQSBcXG90IEEiXSxbMiwyLCJBIl0sWzAsMiwiQSBcXG90IEEiXSxbMCwxLCJcXGV0YSBcXG90IDEiXSxbMSwyLCJcXE9tZWdhIl0sWzAsMywiMSBcXG90IFxcZXRhIiwyXSxbMywyLCJcXE9tZWdhIiwyXSxbMCwyLCJcXGVwc2lsb24iLDJdXQ==
\[\begin{tikzcd}
	A && {A \ot A} \\
	\\
	{A \ot A} && \One
	\arrow["{\eta \ot 1}", from=1-1, to=1-3]
	\arrow["\Omega", from=1-3, to=3-3]
	\arrow["{1 \ot \eta}"', from=1-1, to=3-1]
	\arrow["\Omega"', from=3-1, to=3-3]
	\arrow["\epsilon"', from=1-1, to=3-3]
\end{tikzcd}\]

Such a pairing allows us to give a functor $\RCoMod_A(\cC) \to \RMod_A(\cC)$, which commutes with the fiber functor to $\cC$, endowing $(V, \Delta_V) \in \RCoMod_A(\cC)$ with the module structure
\[
V \ot A \xrightarrow{\Delta_V \ot 1} V \ot A \ot A \xrightarrow{1 \ot \Omega} V \ot \One \cong V.
\]

Notice that this functor commutes with the fiber functor to $\cC$ and preserves free objects.

Conversely, given a functor $F : \RCoMod_A(\cC) \to \RMod_A(\cC)$ which commutes with the fiber functor to $\cC$, we obtain a pairing $\Omega$ on $A$. The functor $F$ yeilds a module structure $\nabla_F : A \ot A \to A$ from the canonical comodule structure on $A$. Then we take
\[
A \ot A \xrightarrow{\nabla_F} A \xrightarrow{\epsilon} \One
\]
as the pairing $\Omega$.

We claim these two constructions are inverse to one another, giving a correspondence between pairings and fiber-preserving fucntors $\RCoMod_A(\cC) \to \RMod_A(\cC)$.

Given a pairing $\Omega$, we produce its associated functor $F$, and then the pairing associated to this functor, which is
\[
A \ot A \xrightarrow{\Delta \ot 1} A \ot A \ot A \xrightarrow{1 \ot \Omega} A \xrightarrow{\epsilon} \One
\]
or,
\[
\epsilon \circ (1 \ot \Omega) \circ (\Delta \circ 1) = \Omega \circ (\epsilon \ot 1 \ot 1) \circ (\Delta \circ 1) = \Omega
\]
where the first equality uses that we are working in a monoidal category and the last uses the counitality axiom.

%in Sweedler's notation, sends
%\[
%a \ot b \mapsto a_{(1)} \ot a_{(2)} \ot b \mapsto a_{(1)}\Omega(a_{(2)}, b) \mapsto \epsilon(a_{(1)})\Omega(a_{(2)}, b) \mapsto \Omega(\epsilon(a_{(1)})a_{(2)}, b) = \Omega(a, b).
%\]

It remains to show that, given $F : \RCoMod_A(\cC) \to \RMod_A(\cC)$, then the functor associated to its associated pairing $\Omega_F$ is simply $F$ itself. We begin by checking this for the object $A$. Observe that the map $A \xrightarrow{\Delta} A \ot A$ is a map of right comodules by the coelagebra axioms. Then under the functor $F$, it becomes a map of right modules. Now, assuming that F preserves free objects, this implies that the following diagram commutes:
% https://q.uiver.app/?q=WzAsNCxbMCwwLCJBIFxcb3QgQSJdLFsyLDAsIkEiXSxbMCwyLCJBIFxcb3QgQSBcXG90IEEiXSxbMiwyLCJBIFxcb3QgQSJdLFswLDEsIlxcbmFibGFfRiJdLFswLDIsIlxcRGVsdGEgXFxvdCAxIiwyXSxbMiwzLCIxIFxcb3QgXFxuYWJsYV9GIiwyXSxbMSwzLCJcXERlbHRhIl1d
\[\begin{tikzcd}
	{A \ot A} && A \\
	\\
	{A \ot A \ot A} && {A \ot A}
	\arrow["{\nabla_F}", from=1-1, to=1-3]
	\arrow["{\Delta \ot 1}"', from=1-1, to=3-1]
	\arrow["{1 \ot \nabla_F}"', from=3-1, to=3-3]
	\arrow["\Delta", from=1-3, to=3-3]
\end{tikzcd}\]

Now, notice that the functor associated to $\Omega_F$ yeilds the following module structure on $A$:
\[
A \ot A \xrightarrow{\Delta \ot 1} A \ot A \ot A \xrightarrow{1 \ot \nabla_F} A \ot A \xrightarrow{1 \ot \epsilon} A
\]
but since the first two arrows are the bottom leg of the above commutative square, we have that this is
\[
(1 \ot \epsilon) \circ \Delta \circ \nabla_F = \nabla_F
\]
using the coalgebra axioms.

We have therefore established a correspondence
\begin{equation*}
        \left\{\begin{array}{c}
        \text{Hopf pairings}\\
        \text{$\Omega : A \ot A \to \One$}
        \end{array}\right\}
        \xleftrightarrow{1:1}
        \left\{\begin{array}{c}
        \text{Functors $\RCoMod_A(\cC) \to \RMod_A(\cC)$}\\
        \text{commuting with fiber functor}\\
        \text{and preserving free objects.}
        \end{array}\right\}.
\end{equation*}

Dually, one can establish a correspondence
\begin{equation*}
        \left\{\begin{array}{c}
        \text{Co-Hopf co-pairings}\\
        \text{$\mho : \One \to A \ot A$}
        \end{array}\right\}
        \xleftrightarrow{1:1}
        \left\{\begin{array}{c}
        \text{Functors $\RMod_A(\cC) \to \RCoMod_A(\cC)$}\\
        \text{commuting with fiber functor}\\
        \text{and preserving free objects.}
        \end{array}\right\}.
\end{equation*}

We will say a Hopf pairing $\Omega$ is nondegenerate if there exists a co-Hopf co-pairing $\mho$ such that $\mho \circ \Omega = \Id_{A \ot A}$ and $\Omega \circ \mho = \Id_{\One}$. Now, it is clear that when $\Omega$ is nonegenerate, the functor $F_\Omega : \RCoMod_A(\cC) \to \RMod_A(\cC)$ is an equivalence.

Since the functors from pairings commute with the fiber functor to $\cC$, it is clear that they are always fully faithful functors. It remains to check essential surjectivity.


DUALLY

We want to say co-pairings $\One \to A \ot A$ correspond to functors $\RMod_A(\cC) \to \RCoMod_A(\cC)$. Given such a functor $F$, it gives $(A, \nabla)$ a comodule structure $\Delta_F$, and induces the co-pairing
\[
\One \xrightarrow{\eta} A \xrightarrow{\Delta_F} A \ot A.
\]
This in turn gives a functor $\RMod_A(\cC) \to \RCoMod_A(\cC)$, which takes $(A, \nabla)$ to the comodule structure
\[
A \xrightarrow{1 \ot \eta} A \ot A \xrightarrow{1 \ot \Delta_F} A \ot A \ot A \xrightarrow{\nabla \ot 1} A \ot A
\]
sending $a \mapsto (\nabla \ot 1)(a \ot \Delta_F(1))$. If $\Delta_F$ is a map of LEFT $A$-modules, then this should suffice to show that going from a functor to its copairing to its functor is the identity.

Why is $\Delta_F$ a map of LEFT $A$-modules?

Recall that $A$ is a right $A$ module, and there is a canonical equivalence $\End(A_A) \cong A$. Now, since $F$ is a functor $\RMod_A(\cC) \to \RCoMod_A(\cC)$, then for any map $\phi : A \to A$ of $A$-modules, this must also be a map of the correspondong comodules under $F$, i.e. the diagram
% https://q.uiver.app/?q=WzAsNCxbMCwwLCJBIl0sWzIsMCwiQSBcXG90IEEiXSxbMCwyLCJBIl0sWzIsMiwiQSBcXG90IEEiXSxbMCwxLCJcXERlbHRhX0YiXSxbMCwyLCJcXHBoaSIsMl0sWzIsMywiXFxEZWx0YV9GIiwyXSxbMSwzLCJcXHBoaSBcXG90IDEiXV0=
\[\begin{tikzcd}
	A && {A \ot A} \\
	\\
	A && {A \ot A}
	\arrow["{\Delta_F}", from=1-1, to=1-3]
	\arrow["\phi"', from=1-1, to=3-1]
	\arrow["{\Delta_F}"', from=3-1, to=3-3]
	\arrow["{\phi \ot 1}", from=1-3, to=3-3]
\end{tikzcd}\]
commutes. But taking $\phi$ to be left-multiplication in $A$, which  we can do by the above isomorphism, we see that commutativity of the square says $\Delta_F$ is a map of left $A$-modules.

So we see that $F$ is recovered for $A$. Now let us show that it is recovered for any module $M$.

Given any map $A \xrightarrow{f} M$ of right $A$-modules, under $F$ it becomes a map of right $A$-comodules, so that
\[\begin{tikzcd}
	A && {A \ot A} \\
	\\
	M && {M \ot A}
	\arrow["{\Delta_F}", from=1-1, to=1-3]
	\arrow["f"', from=1-1, to=3-1]
	\arrow["{\Delta_F}"', from=3-1, to=3-3]
	\arrow["{f \ot 1}", from=1-3, to=3-3]
\end{tikzcd}\]
commutes. Then letting $f$ be the map sending $1_A \mapsto m \in M$, we have that $\Delta_F(m) = (f \ot 1)\Delta_F(1_A)$. So the comodule structure under $F$ for $M$ is determined by the comodule strucutre for $A$, and it follows that we have preservation for modules.

\end{appendices}

\printbibliography

\end{document}
