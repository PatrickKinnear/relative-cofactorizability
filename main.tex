\documentclass{article}
\usepackage[utf8]{inputenc}

\usepackage{hyperref}

\usepackage{quiver}
\usepackage{pkinne}

\usepackage{tikz}
\usepackage{ifthen}
\usepackage{intcalc}
\usetikzlibrary{positioning,calc}

\newcommand{\stackspace}{4}
\newcommand{\stackalternate}[2][1cm]{\;\tikz[baseline, yshift=.65ex]%
    {\foreach \k [evaluate=\k as \r using (.5*#2+.5-\k)*\stackspace] in {1,...,#2}{%
    \ifodd\k{\draw[->](0,\r pt)--(#1,\r pt);}%
    \else{\draw[<-](0,\r pt)--(#1,\r pt);}\fi
    }}\;}
\newcommand{\stackrightarrow}[2][1cm]{\;\tikz[baseline, yshift=.65ex]%
    {\foreach \k [evaluate=\k as \r using (.5*#2+.5-\k)*\stackspace] in {1,...,#2}{%
    \draw[->](0,\r pt)--(#1,\r pt);%
    }}\;}
\newcommand{\stackleftarrow}[2][1cm]{\;\tikz[baseline, yshift=.65ex]%
    {\foreach \k [evaluate=\k as \r using (.5*#2+.5-\k)*\stackspace] in {1,...,#2}{%
    \draw[<-](0,\r pt)--(#1,\r pt);
    }}\;}

\title{The Invertibility Conjecture}
\author{Patrick Kinnear}
\date{}

\usepackage{biblatex}
\addbibresource{invertibility-conjecture.bib}

\begin{document}

The invertibility conjecture is that $\Rep_q(G)$ is
invertible as a 1-morphism $\Rep(G) \to \Rep(G)$ in the Morita category $\Alg_3(\Pr)$.

The following theorem characterises invertible 1-morphisms in $\Alg_3(\Pr)$.

\begin{thm}
Let $\cA, \cB$ be $E_3$-algebras in $\cS$, and $\cC$ a morphism $\cA \to \cB$. That is, $\cC$ is an $E_2$-algebra in $(\cA, \cB)$-bimodules, or equivalently, an $E_2$-algebra in $\cS$ equipped with a braided tensor functor $\cA \bt \cB^{3 \op} \to Z_2(\cC)$. Then $\cC$ is invertible, considered as a morphism of $E_3$-algebras, if and only if it is dualizable as a module over $\cA, \cB, \cC^e_\cA := \cC^{\ot \op} \bt_\cA \cC, \cC^e_\cB := \cC \bt_\cB \cC^{\ot \op}, \cC^{\sigma \op} \bt_\cA \cC, \cC \bt_\cB \cC^{\sigma \op}, HC_\cA(\cC) := \cC \bt_{\cC^{\sigma \op} \bt_\cA \cC} \cC^{\ot \op}$, and $HC_\cB(\cC) := \cC \bt_{\cC \bt_\cB \cC^{\sigma \op}} \cC^{\ot \op}$ and moreover the following maps are isomorphisms:
\begin{enumerate}
        \item $HC_\cA(\cC) \to \Hom_\cB(\cC, \cC)$.
        \item $\cC^{\sigma \op} \bt_\cA \cC \to \Hom_{\cC^e_\cB}(\cC, \cC)$.
        \item $\cC \bt_\cB \cC^{\ot \op} \to \Hom_{\cC^{\sigma \op} \bt_\cA \cC}(\cC, \cC)$
        \item $\cB \to \Hom_{HC_\cA(\cC)}(\cC, \cC)$.
        \item $HC_\cB(\cC) \to \Hom_\cA(\cC, \cC)$.
        \item $\cA \to \Hom_{HC_\cB(\cC)}(\cC, \cC)$.
\end{enumerate}
\end{thm}

We therefore need to be able to compute expressions such as $\Rep_q(G)^{\sigma \op} \bt_{\Rep(G)} \Rep_q(G)$.

We \pk{note that} the functors in the conditions above are braided tensor functors \pk{!} so it suffices to show they are equivalences at the level of plain categories (we may ignore $\sigma \op, \ot \op$ etc).

Moreover, in the case we are considering, that maps 4 and 6 are equivalences is well-known.

We begin by attempting to compute $\Rep_q(G) \bt_{\Rep(G)} \Rep_q(G)$. Recall that this is given by the colimit of the diagram below.

\[
\Rep_q(G) \bt \Rep_q(G) \stackleftarrow{2} \Rep_q(G) \bt \Rep(G) \Rep_q(G) \stackleftarrow{3} \Rep_q(G) \bt \Rep(G) \bt \Rep(G) \bt \Rep_q(G) \stackleftarrow{4} \dots
\]

Now, the following theorem is implicit in Lurie's Higher Algebra but is recounted in an accessible form by Gaitsgory.

\begin{thm}[{\cite[Cor. C.2.3]{gaitsgorySheavesCategoriesNotion2015}}]
Let $\cA$ be a monoidal category acting on $\cC$ on the right and on $\cD$ on the left. If either: the two action functors and the monoidal product admit right adjoints, and the right adjoint to the monoidal product is a map of $\cA$-module categories; or the two action functors and the monoidal product admit left adjoints, and the left adjoint to the monoidal product is a map of $\cA$-module categories; then the map
\[
\cC \bt \cD \to \cC \bt_\cA \cD
\]
admits a right adjoint and the adjunction is monadic, with the monad isomorphic as a plain functor to
\[
(\mathrm{act}_\cC \ot \Id) \circ (\Id \ot \mathrm{act}_\cD)^R.
\]
\end{thm}

Then if we can verify the conditions of this theorem, this will give a monadic description of the relative tensor product. The monoidal multiplication of $\Rep(G)$ has a right adjoint $X \mapsto X \bt X$ (universal property of tensor products of representations).

We would like to show that the map

\begin{align*}
\Rep_q(G) \bt \Rep(G) &\to \Rep_q(G)\\
V \bt W &\mapsto V \ot Fr^*(W)
\end{align*}

has a right adjoint. This should be something like the map
\[
X \mapsto X^{u_q(G)}
\]
or
\[
X \mapsto X/X^{u_q(G)}
\]
where $u_q(G) \subseteq \Ures_q$ is the small quantum group and we expect there is an adjunction $(Fr^* \dashv -^{u_q(G)})$ or $(Fr^* \dashv -/-^{u_q(G)})$. If this is the case, then as we claim the right action has a right adjoint (and similarly, so does the left action).

Supposing we have dealt with this, we then have that $\Rep_q(G) \bt_{\Rep(G)} \Rep_q(G)$ is equivalent to algebras for the monad
\[
(\mathrm{act}_\cC \ot \Id) \circ (\Id \ot \mathrm{act}_\cD)^R.
\]
on $\Rep_q(G) \bt \Rep_q(G)$.



\printbibliography

\end{document}
