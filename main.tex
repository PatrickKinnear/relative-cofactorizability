\documentclass{article}
\usepackage[utf8]{inputenc}

\usepackage{hyperref}

\usepackage{quiver}
\usepackage{pkinne}

\usepackage{tikz}
\usepackage{ifthen}
\usepackage{intcalc}
\usetikzlibrary{positioning,calc}

\newcommand{\stackspace}{4}
\newcommand{\stackalternate}[2][1cm]{\;\tikz[baseline, yshift=.65ex]%
    {\foreach \k [evaluate=\k as \r using (.5*#2+.5-\k)*\stackspace] in {1,...,#2}{%
    \ifodd\k{\draw[->](0,\r pt)--(#1,\r pt);}%
    \else{\draw[<-](0,\r pt)--(#1,\r pt);}\fi
    }}\;}
\newcommand{\stackrightarrow}[2][1cm]{\;\tikz[baseline, yshift=.65ex]%
    {\foreach \k [evaluate=\k as \r using (.5*#2+.5-\k)*\stackspace] in {1,...,#2}{%
    \draw[->](0,\r pt)--(#1,\r pt);%
    }}\;}
\newcommand{\stackleftarrow}[2][1cm]{\;\tikz[baseline, yshift=.65ex]%
    {\foreach \k [evaluate=\k as \r using (.5*#2+.5-\k)*\stackspace] in {1,...,#2}{%
    \draw[<-](0,\r pt)--(#1,\r pt);
    }}\;}

\title{The Invertibility Conjecture}
\author{Patrick Kinnear}
\date{}

\usepackage{biblatex}
\addbibresource{invertibility-conjecture.bib}

\newcommand\tens{\mathrm{tens}}

\begin{document}

\maketitle

The invertibility conjecture is that $\Rep_q(G)$ is
invertible as a 1-morphism $\Rep(G) \to \Rep(G)$ in the Morita category $\Alg_3(\Pr)$.

The following theorem characterises invertible 1-morphisms in $\Alg_3(\Pr)$.

\begin{thm}
Let $\cA, \cB$ be $E_3$-algebras in $\cS$, and $\cC$ a morphism $\cA \to \cB$. That is, $\cC$ is an $E_2$-algebra in $(\cA, \cB)$-bimodules, or equivalently, an $E_2$-algebra in $\cS$ equipped with a braided tensor functor $\cA \bt \cB^{3 \op} \to Z_2(\cC)$. Then $\cC$ is invertible, considered as a morphism of $E_3$-algebras, if and only if it is dualizable as a module over $\cA, \cB, \cC^e_\cA := \cC^{\ot \op} \bt_\cA \cC, \cC^e_\cB := \cC \bt_\cB \cC^{\ot \op}, \cC^{\sigma \op} \bt_\cA \cC, \cC \bt_\cB \cC^{\sigma \op}, HC_\cA(\cC) := \cC \bt_{\cC^{\sigma \op} \bt_\cA \cC} \cC^{\ot \op}$, and $HC_\cB(\cC) := \cC \bt_{\cC \bt_\cB \cC^{\sigma \op}} \cC^{\ot \op}$ and moreover the following maps are isomorphisms:
\begin{enumerate}
        \item $HC_\cA(\cC) \to \Hom_\cB(\cC, \cC)$.
        \item $\cC^{\sigma \op} \bt_\cA \cC \to \Hom_{\cC^e_\cB}(\cC, \cC)$.
        \item $\cC \bt_\cB \cC^{\ot \op} \to \Hom_{\cC^{\sigma \op} \bt_\cA \cC}(\cC, \cC)$
        \item $\cB \to \Hom_{HC_\cA(\cC)}(\cC, \cC)$.
        \item $HC_\cB(\cC) \to \Hom_\cA(\cC, \cC)$.
        \item $\cA \to \Hom_{HC_\cB(\cC)}(\cC, \cC)$.
\end{enumerate}
\end{thm}

After \cite{brochierInvertibleBraidedTensor2020}, we refer to conditions 1 and 5 as relative cofactorizability, conditions 2 and 3 as relative factorizability, and conditions 4 and 6 as relative non-degeneracy. We will treat these in turn.

We \pk{note that} the functors in the conditions above are braided tensor functors \pk{!} so it suffices to show they are equivalences at the level of plain categories (we may ignore $\sigma \op, \ot \op$ etc).

\section{Relative non-degeneracy}

Since our background category $\Pr$ is a 2-category, we can establish that $\Hom_{HC_\cA(\cC)}(\cC, \cC) \simeq Z_2(\cC)$ \pk{give the argument! Uses ideas such as those in \cite{laugwitzRelativeMonoidalCenter2020, ben-zviQuantumCharacterVarieties2018}}.

Then it is understood that $\Rep(G) \simeq Z_2(\Rep_q(G))$. See \cite{ganevQuantumFrobeniusCharacter2020, negronLogmodularQuantumGroups2021}.

This establishes relative non-degeneracy.

\section{Relative factorizability}

%We therefore need to be able to compute expressions such as $\Rep_q(G)^{\sigma \op} \bt_{\Rep(G)} \Rep_q(G)$.

We begin with the following computation of $\Rep_q(G) \bt_{\Rep(G)} \Rep_q(G)$.

\begin{prop}
\label{p-Rep_q_Rep_Rep_q-is-O(G)-mod}
There is an equivalence
\[
\Rep_q(G) \bt_{\Rep(G)} \Rep_q(G) \simeq \cO(G)-\mod(\Rep_q(G) \bt \Rep_q(G))
\]
such that the tensor product functor $T : \Rep_q(G) \bt_{\Rep(G)} \Rep_q(G) \to \Rep_q(G)$ is equivalent to the functor
\[
a \bt b \mapsto a \bt b \bt_{\cO(G)} k
\]
\end{prop}

We define $o_q(G) = TT^R(\One) \in \Rep_q(G)$.

\begin{prop}
\label{p-factorizability-equivs}
We have equivalences
\begin{enumerate}
\item \label{eq-Rep_q_Rep_Rep_q-is-o_q-comod} $\Rep_q(G) \bt_{\Rep(G)} \Rep_q(G) \simeq o_q(G)-comod(\Rep_q(G))$,
\item \label{eq-rel-Z_1-is-o_q-mod} $\Hom_{\Rep_q(G)^e_{\Rep(G)}}(\Rep_q(G)) \simeq o_q(G)-mod(\Rep_q(G))$.
\end{enumerate}
\end{prop}

There exists a \pk{canonical} pairing $\Omega : o_q(G) \ot o_q(G) \to \One$. Under such a pairing, the factorizability map becomes
\[
o_q(G)-comod(\Rep_q(G)) \xrightarrow{\Omega} o_q(G)-mod(\Rep_q(G))
\]
which is an equivalence if and only if $\Omega$ is nondegenerate. Therefore, relatie factorizability is established if we can show the following.

\begin{prop}
\label{p-Omega-nondegenerate}
The pairing $\Omega : o_q(G) \ot o_q(G) \to \One$ is nondegenerate.
\end{prop}

\begin{proof}
We notice that, under the forgetful functor $\Rep_q(G) \to \Rep u_q(G)$ induced by $u_q(G) \hookrightarrow \Ures_q(G)$, we have $o_q(G)$ is the canonical coend of \cite{brochierInvertibleBraidedTensor2020} \pk{why?}. Then, for $q$ \pk{sufficiently nice}, $\Rep u_q(G)$ is an invertible finite tensor category \cite[Rmk. 3.12]{brochierInvertibleBraidedTensor2020} hence the pairing $\Omega$ is non-degenrate by \cite[Thm. 3.20]{brochierInvertibleBraidedTensor2020}.
\end{proof}
Therefore, the relative factorizability conditions hold.

Below we sketch the proof of Proposition \ref{p-Rep_q_Rep_Rep_q-is-O(G)-mod}.

Recall that $\Rep_q(G) \bt_{\Rep(G)} \Rep_q(G)$ is given by the colimit of the diagram below.

\[
\Rep_q(G) \bt \Rep_q(G) \stackleftarrow{2} \Rep_q(G) \bt \Rep(G) \Rep_q(G) \stackleftarrow{3} \Rep_q(G) \bt \Rep(G) \bt \Rep(G) \bt \Rep_q(G) \stackleftarrow{4} \dots
\]

Now, the following theorem is implicit in Lurie's Higher Algebra but is recounted in an accessible form by Gaitsgory.

\begin{thm}[{\cite[Cor. C.2.3]{gaitsgorySheavesCategoriesNotion2015}}]
Let $\cA$ be a monoidal category acting on $\cC$ on the right and on $\cD$ on the left. If either: the two action functors and the monoidal product admit continuous right adjoints, and the right adjoint to the monoidal product is a map of $\cA$-module categories; or the two action functors and the monoidal product admit left adjoints, and the left adjoint to the monoidal product is a map of $\cA$-module categories; then the map
\[
\cC \bt \cD \to \cC \bt_\cA \cD
\]
admits a right adjoint and the adjunction is monadic, with the monad isomorphic as a plain functor to
\[
(\mathrm{act}_\cC \ot \Id) \circ (\Id \ot \mathrm{act}_\cD)^R.
\]
\end{thm}

Then if we can verify the conditions of this theorem, this will give a monadic description of the relative tensor product.

By \cite[Definition-Proposition 4.1]{brochierDualizabilityBraidedTensor2018}, if $\Rep(G)$ is cp-rigid then its multiplication admits a cocontinuous right adjoint, e.g. an adjoint in $\Pr$. To be cp-rigid means all compact projectives are left and right dualizable. Projective means the existence of a ``dual basis", does being compact imply this is finite? Then any category $R$-mod is cp-rigid...

\pk{How to show this right adjoint is continuous?}

%The monoidal multiplication of $\Rep(G)$ has a right adjoint $X \mapsto X \bt X$ (universal property of tensor products of representations).

We would like to show that the map

\begin{align*}
\mathrm{act}_{\Rep_q(G)} : \Rep_q(G) \bt \Rep(G) &\to \Rep_q(G)\\
V \bt W &\mapsto V \ot Fr^*(W)
\end{align*}

has a right adjoint, given by
\[
\mathrm{act}^R(Y) = \int^{X \in \Rep^{\mathrm{fd}}(G)}(Y \ot Fr^*(X^\vee)) \boxtimes X.
\]
The proof is analogous to Section 1.2 of \cite{kalmykovCategoricalApproachDynamical2020} \pk{The key is to understand Prop. 1.5}. Since it's a coend, the right adjoint preserves colimits so exists in $\Pr$. \pk{what about continuity?}

%This should be something like the map
%\[
%X \mapsto X^{u_q(G)}
%\]
%or
%\[
%X \mapsto X/X^{u_q(G)}
%\]
%where $u_q(G) \subseteq \Ures_q$ is the small quantum group and we expect there is an adjunction $(Fr^* \dashv -^{u_q(G)})$ or $(Fr^* \dashv -/-^{u_q(G)})$. If this is the case, then as we claim the right action has a right adjoint (and similarly, so does the left action).

Supposing we have dealt with this, we then have that $\Rep_q(G) \bt_{\Rep(G)} \Rep_q(G)$ is equivalent to algebras for the monad
\[
(\mathrm{act}_\cC \ot \Id) \circ (\Id \ot \mathrm{act}_\cD)^R.
\]
on $\Rep_q(G) \bt \Rep_q(G)$.

\section{Relative cofactorizability}
We echo the argument of \cite{brochierInvertibleBraidedTensor2020} that cofactorizability is equivalent to factorizability. Recall the functor
\begin{align*}
\tens: \cC &\to \Fun(\cC, \cC)\\
x &\mapsto x \ot -
\end{align*}

We begin with the following claim.

\begin{prop}
The functor $\tens$ is the dual of $T : \cC \bt \cC \to \cC$ as a map of $\cC_1$-modules (here $\cC_1$ denoted the left factor of $\cC$ in $\cC \bt \cC$).
\end{prop}

\begin{proof}(Idea) Clearly the left copy of $\cC$ is left-dualizable as a $\cC$-module. So we have
\begin{align*}
\Hom(\cC \bt \cC, \cC) &\simeq \Hom(\cC, \cC^{\vee} \bt \cC)\\
f &\mapsto (1 \ot f) \circ (coev' \ot 1)
\end{align*}
and we have
\[
T \mapsto (c \mapsto (\sum c^i \bt c_i) \bt c \mapsto \sum c^i \bt (c_i \ot c))
\]
where we see that the right functor is simply $\tens$. \pk{possibly on the right?}
\end{proof}

Then it follows that:

\begin{prop}
$T$ has a left adjoint which is a functor of $\cC \bt \cC$-module categories if and only if $\tens$ has a right adjoint.
\end{prop}

\begin{proof}
The idea is that, $(T^{\vee})^R = (T^L)^{\vee}$. \pk{A bit muddled on this!}
\end{proof}

We \pk{claim} that in our case $T$ has a left adjoint, so $\tens$ has a right adjoint, and we define $E = \tens^R(\One)$.

We can also argue the following.

\begin{prop}
There is an equivalence
\[
HC_{\cA}(\cC) \simeq F-mod(\cC).
\]
\end{prop}

\begin{proof}
Let $T : \Rep_q(G) \bt \Rep_q(G) \to \Rep_q(G)$ the tensor product functor. \pk{Recall} that $\Rep_q(G) \simeq T^R(\One)-mod(\Rep_q(G) \bt \Rep_q(G))$. Now, it is a \pk{fact} that
\[
A-mod(\cC) \bt_\cC \cM \simeq A-mod(\cM)
\]
(\pk{what conditions here?}) where $\cM$ is a $\cC$-module category. Here, we would like to take $\cC = \Rep_q(G) \bt_{\Rep(G)} \Rep_q(G) \simeq \cO(G)-\mod(\Rep_q(G) \bt \Rep_q(G))$, $\cM \simeq \Rep_q(G)$, and somehow we'd like $A-mod(\cC) \simeq \Rep_q(G)$. Then the displayed identity will give $HC_{\Rep(G)}(\Rep_q(G)) \simeq A-mod(\Rep_q(G))$.
\end{proof}

This is related to Prop \ref{p-Rep_q_Rep_Rep_q-is-O(G)-mod}.

Now, there exists a map $Dr: F \to E$ which should be related to $\Omega$ by tensor-hom. Under the above identifications, the relative cofactorizability becomes
\begin{align*}
F-mod(\cC) &\to E-mod(\cC)\\
M &\mapsto M \ot_F E
\end{align*}
along $Dr$. But this is an equivalence if and only if $Dr$ is an isomorphism, if and only if $\Omega$ is non-degenerate, which it is by Proposition \ref{p-Omega-nondegenerate}.

Then this establishes relative cofactorizability.

\printbibliography

\end{document}
